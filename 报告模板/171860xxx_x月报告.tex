% !TEX program = pdflatex
\documentclass[a4paper,UTF8]{article}
\usepackage{ctex}
\usepackage[margin=1.25in]{geometry}
\usepackage{color}
\usepackage{graphicx}
\usepackage{amssymb}
\usepackage{amsmath}
\usepackage{amsthm}
%\usepackage[thmmarks, amsmath, thref]{ntheorem}
\theoremstyle{definition}
\newtheorem*{solution}{Solution}
\newtheorem*{prove}{Proof}
\usepackage{multirow}
\usepackage{url}
\usepackage[colorlinks,urlcolor=blue]{hyperref}
\usepackage{enumerate}
\usepackage{cite}
\renewcommand\refname{参考文献}


%--

%--
\begin{document}
\title{\textbf{《计算机图形学》九月报告}}
\author{181860127   姚逸斐  \href{mailto:181860127@smail.nju.edu.cn}{181860127@smail.nju.edu.cn}}
\maketitle

\section{综述}
/*进度报告/系统报告要求
•	在上个月报告的基础上添加本月新的内容即可,无需从头重写
•	报告内容包括:
o	已完成或拟采用算法的原理介绍、自己的理解、对比分析等
o	已完成或拟采用的系统框架、交互逻辑、设计思路等
o	介绍自己系统中的巧妙的设计、额外的功能、易用的交互、优雅的代码、好看的界面等(可选)
•	请附上联系方式(邮箱或QQ等),以便出现问题时及时联系
•	需注明在实现作业过程中使用的参考资料,包括技术博客等
•	可添加附加材料(觉得需要附加说明的代码等)*/

\section{算法介绍}
\subsection{直线}

\subsubsection{DDA算法}
\textbf{原理介绍}\par
\cite{GraffZivin2018}
\textbf{个人理解}\par

\textbf{对比分析}\par
ss
\subsubsection{Bresenham算法}
\textbf{原理介绍}\par
\cite{GraffZivin2018}
\textbf{个人理解}\par

\textbf{对比分析}\par

\dots
\subsection{多边形}
类似直线
\subsection{椭圆}
\subsubsection{中点圆生成算法}
\textbf{原理介绍}\par

\textbf{个人理解}\par

\textbf{对比分析}\par
\subsection{曲线}
\subsubsection{Bezier算法}
\textbf{原理介绍}\par

\textbf{个人理解}\par

\textbf{对比分析}\par
\subsubsection{B-spline算法}
\textbf{原理介绍}\par

\textbf{个人理解}\par

\textbf{对比分析}\par
\section{系统介绍}
\dots

\section{总结}
\dots

%\bibliographystyle{plain}%
%"xxx" should be your citing file's name.
%\bibliography{info}

\bibliography{info}
\bibliographystyle{plainnat}


\end{document}